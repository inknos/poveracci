
\documentclass[10pt,twoside,twocolumn]{article}
\usepackage[bg-letter]{dnd} % Options: bg-a4, bg-letter, bg-full, bg-print, bg-none.
\usepackage[english]{babel} % Trennungsregeln und autom. Überschriften in n. dt. RS
\usepackage[utf8]{inputenc}

\usepackage{hyperref}


\title{D\&D Campaign}
\author{Nicola Sella}
\date{\today}

% Start document
\begin{document}

\maketitle

\tableofcontents

\fontfamily{ppl}\selectfont % Set text font

\section{Isola Misteriosa}
Non si sapeva nulla dell'isola trovata poco a sud di Blasingdell.
\`E stata scoperta una antica civilt\`a che abitava anticamente il posto.
All'interno di un vulcano \`e stato trovato un tempio con un percorso di
purificazione dai mali e i crimini dell'uomo contro la natura. Una figura si \`e
rivelata alla fine di tutto dicendo di essere la protettrice del luogo.

\subsection{Scritte trovate all'interno}
\textit{
  Nelle viscere della montagna\\
  pulsa il cuore della terra:\\
  solo il fuoco pu\`o purgare\\
  i mali dell'uomo\\
  che in terra regna.\\
  \\
  Le fiamme purificano ci\`o che l'uomo rende impuro.\\
  \\
  Carezzami leggera sulle ferite.\\
  Lenitivo che non cura.\\
  Sei un gentile abbraccio sul mondo, oh dura sorella crudele.\\
  \\
  Vorticoso scendi\\
  e lasciati travolgere nel fondo.\\
  Non ascoltare più\\
  fingi di esser cieco e di esser sordo.\\
  Resta dentro il buio\\
  Dormi e scendi in questo girotondo.\\
  \\
  Soffri perché la sincerità fa male. La terra trema quando cerca di guarire.\\
  Ridi e non pensare: menti a te stesso e continua a camminare.\\
  \\
  Il destino è l'amara sorte che accompagna i viventi.\\
  \\
  Sgorga, da profonde ferite, il sangue della terra: sangue che ribolle, soffoca, brucia.\\
  Senti il peso di queste viscere che schiacciano il suolo. Inutile sacco: otre pieno di calunnie\\
  \\
  Delle peggiori voglie umane\\
  La più meschina è la bramosia di divorare\\
  impunemente i frutti della terra.\\
  \\
  Convinci te stesso che questa è la FINE.\\
  Ormai è tardi per seminare la terra.\\
  Convinci te stesso che questa è la FINE.\\
  Si è sempre in tempo per seminare un'idea.\\
  Convinci te stesso che questa è la fine.\\
  \\
  In questa terra polverosa nessun gaudio rimane.\\
  Nulla, non una parola di conforto.\\
  Rallegra solo la folle illusione,\\
  per nascondere algi occhi questo grande aborto,\\
  di una vana speranza:\\
  un amaro sorriso\\
  sul viso scempiato\\
  del torturatore colpevole\\
  in questa stanza.\\
}
\section{Gli elfi di Estor}
Estor \`e una della quattro maggiori isole nel mondo di Halaia. \`E abitata
principalmente da elfi anche se nelle capitali la popolazione \`e omogenea.
L'isola \`e suddivisa da una catena montuosa in una parte nord e sud: Estor
\footnote{Estor elfico per grande regno}
e Umberul\footnote{Elfico per regno notturno}. La capitale si trova a nord
ed \`e chiamata Mithlandris\footnote{Principe che corre: nota che principe in elfico
non ha genere maschile ma \`e abbinato sia ai principi che alle principesse}; a sud
vi \`e Mithlandor\footnote{Principe della notte}, citt\`a gemella
della capitale e principale capoluogo della regione.
Altre due citt\`a sono molto importanti: Menyamar e Meyrador
\footnote{La radice della parola fa rifermento al fratello minore del
  principe, per cui non esiste un equivalente nella lingua comune.
  Rispettivamente possono essere tradotte come principino delle montagne
  e principino della notte}.\\
Niirith Omlir, sorella di Fahlsias Omlir, governa il regno Mithlandris, mentre
il fratello minore governa il regno inferiore.
Niirith \`e sposata con Throdiore Naemaris, con il quale ha due figli. Il pi\`u
grande, Alagor, governa la citt\`a di Meyrador insieme alla sua consorte Elmilin.
La minore si chiama Minireth ed \`e una guaritrice famosa nel regno.\\
Fahlsias \`e sposato con Alotel Adanodel: Nelalwe \`e la figlia maggiore della
giovane coppia elfica. Poco si sa sul conto di Nelalwe, il suo animo avventuroso la
ha spinta a viaggiare da sempre: ha infatti abbandonato la casa del padre poco
dopo aver raggiunto la sua maggiore et\`a.
Indithrel, secondogenita gemella di Nelalwe, regge la citt\`a di Menyamar insieme
al suo consorte Lebrandil, un elfo del basso ceto che per\`o ha dimostrato grandi
doti militari.
Findore \`e il fratello pi\`u piccolo. \`E ossessionato dalla scomparsa della sorella
Nelalwe e vuole seguirla non appena raggiunta la maggiore et\`a.\\
Il governo elfico \`e spesso affiancato da famiglie importanti che danno consiglio
ai regnanti. Delegal Ceruleadel e Elen aiutano i sovrani di Mithlandris nella loro
reggenza. I tre figli di questa coppia sono ormai famosi nelle isole per essere
rispettivamente una brava commerciante, una alchimista potente e un astronomo:
sono Vadania, Liia e Amiel. Amiel \`e famoso anche per la sua promessa sposa, un'umana.\\
Il trono di Mithlandor ha come consiglieri Felosial Guernodel e Deroua Galothel.
Questi non hanno figli.

\section{Personaggi}
\subsection{Ulfe}
Ogre capitoano del manipolo di orchi dentro lo stone tooth.
\subsection{Gerey Horne}
Uno dei capi della guardia di Blasingdell. Amico di Undecimus.
\subsubsection{Cenbert, Hele}
Guardie di Blasingdell amiche di Undecimus.
\subsection{Herumbr, Alin, Linda}
Tre nomi misteriosi di possibili halfling trovati all'interno
dello stone tooth con un contratto da parte di Aebeard V.
Trovati due corpi su tre.
\subsection{Sige}
Locandiere di Whyam. Da il contratto delle lame di Durgeddin.
\subsection{Wol}
Locandiere di Blasingdell. In stretto contatto con Sige e Arwebrias
\subsection{Sir Miles Berrick}
Reggente di Blasingdell, pare sia un sovrano giusto e attento alla gente
seppur molto preso da questioni personali ed economiche.
\subsection{Snurrevin}
Duergar superstite dal raid dello stone tooth. Catturato dalle guardie di
Blasingdell dipo la sua fuga dalla montagna \`e stato imprigionato in una
nave piena di pirati e mercenari per essere interrogato. Pare che le
sue informazioni riguardo alla posizione strategica della montagna
potessero fare gola a molte bocche.
\subsubsection{Brawngnaw}
Fedele topo di Snurrevin, assassinato malamente davanti agli occhi del padrone.
\subsection{Bryaney Pethey}
Comandante della guardia cittadina di Blasingdell. Uomo compromesso. Si dice che
sia un uomo giusto ma sia troppo coinvolto dal suo lavoro e la sua moralit\`a
sia per questo motivo compromessa. I pareri su di lui sono discordanti. Geoffrey
Lo accusa, Gerey lo difende.
\subsection{Liamond}
Ricercato nel quartiere mercantile di Blasingdell come un capo mafia di una
lunga serie di furtarelli di citt\`a.
\subsection{Shiishtel}
Spietato combattente di arene. Yuan ti che combatte per fama e denaro, rispetta
solo le persone che secondo lui possono meritarsi la sua attenzione. Ha risparmiato una
sola volta no sfidante nella sua vita: Ralphia.
\subsection{Ralphia}
Conosce Arwebrias, Risparmiato da Shiistel.
\subsection{Aebeard Verne}
Misterioso Commerciante della zona del porto. Era stato indicato da Gerey come un
aiuto per risolvere il mistero della nave vuota. Fuori citt\`a quando \`e stato cercato.
\subsection{Capitano Alan: 12 anime}
Famoso pirata e predone dei mari del sud. Responsabile anche del rapimento di Nanande e
del massacro dei suoi compagni. Mercenario ambizioso e senza scrupoli assoldato
dalla guardia cittadina per sorvegliare Snurrevin. Collegato in qualche modo alle
Ombre Del Sole.
\subsubsection{Gorbal Tuhurk}
Nostromo e fedele braccio destro di Alan. 
\subsubsection{Barion Gilli Strongbeard}
Mercenario di provenienza ignota al servizio di Alan.
\subsubsection{Ezzel'n Shii}
Vedetta di Alan.
\subsection{Geoffrey/Athuith}
Mercante di blasingdel e trafficante al porto, conosciuto con un doppio nome.
Sembre che abbia interessi simili a quelli di Arwebrias anche se si occupano di
cose diverse.
\subsection{Arwebrias Galothel}
Elfa ricercatissima in citt\`a per la sua fama dovuta probabilmente alla sua identit\`a
molto ben celata e alla sua difficolt\`a nel lasciare tracce o di farsi catturare.
Pare che sappia molte pi\`u cose di quelle che dice. Sa molte cose di politica e difende
degli ideali di libert\`a. Non le piace Athuith ma lo rispetta come fuorilegge. \`E al
momento collegata con tutti gli incarichi svolti finora, o come mandante o per complicit\`a
e coinvolgimento.
\end{document}