
\documentclass[10pt,twoside,twocolumn]{article}
\usepackage[bg-letter]{dnd} % Options: bg-a4, bg-letter, bg-full, bg-print, bg-none.
\usepackage[english]{babel} % Trennungsregeln und autom. Überschriften in n. dt. RS
\usepackage[utf8]{inputenc}

\usepackage{hyperref}
\title{D\&D Campaign}
\author{Nicola Sella}
\date{\today}

% Start document
\begin{document}

\maketitle

\tableofcontents

\fontfamily{ppl}\selectfont % Set text font
\section{Avvenimenti di trama}
\subsection{Guerra tra i regni}
Ombre del Sole sono interessate a compiere un colpo di stato per rovesciare il regno degli umani e
poi degli elfi e prendere il potere. 

\subsection{Fonte di potere isola misteriosa}
Kaligribur \`e interessato a trovare la fonte di potere che potrebbe essere in un'isola
inesplorata dei mari.
\section{Isola Misteriosa}
Non si sapeva nulla dell'isola trovata poco a sud di Blasingdell.
\`E stata scoperta una antica civilt\`a che abitava anticamente il posto.
All'interno di un vulcano \`e stato trovato un tempio con un percorso di
purificazione dai mali e i crimini dell'uomo contro la natura. Una figura, Quel' Cha \ref{char:quelcha}, si \`e
rivelata alla fine di tutto dicendo di essere la protettrice del luogo.

\subsection{Scritte trovate all'interno}
\textit{
  Nelle viscere della montagna\\
  pulsa il cuore della terra:\\
  solo il fuoco pu\`o purgare\\
  i mali dell'uomo\\
  che in terra regna.\\
  \\
  Le fiamme purificano ci\`o che l'uomo rende impuro.\\
  \\
  Carezzami leggera sulle ferite.\\
  Lenitivo che non cura.\\
  Sei un gentile abbraccio sul mondo, oh dura sorella crudele.\\
  \\
  Vorticoso scendi\\
  e lasciati travolgere nel fondo.\\
  Non ascoltare più\\
  fingi di esser cieco e di esser sordo.\\
  Resta dentro il buio\\
  Dormi e scendi in questo girotondo.\\
  \\
  Soffri perché la sincerità fa male. La terra trema quando cerca di guarire.\\
  Ridi e non pensare: menti a te stesso e continua a camminare.\\
  \\
  Il destino è l'amara sorte che accompagna i viventi.\\
  \\
  Sgorga, da profonde ferite, il sangue della terra: sangue che ribolle, soffoca, brucia.\\
  Senti il peso di queste viscere che schiacciano il suolo. Inutile sacco: otre pieno di calunnie\\
  \\
  Delle peggiori voglie umane\\
  La più meschina è la bramosia di divorare\\
  impunemente i frutti della terra.\\
  \\
  Convinci te stesso che questa è la FINE.\\
  Ormai è tardi per seminare la terra.\\
  Convinci te stesso che questa è la FINE.\\
  Si è sempre in tempo per seminare un'idea.\\
  Convinci te stesso che questa è la fine.\\
  \\
  In questa terra polverosa nessun gaudio rimane.\\
  Nulla, non una parola di conforto.\\
  Rallegra solo la folle illusione,\\
  per nascondere algi occhi questo grande aborto,\\
  di una vana speranza:\\
  un amaro sorriso\\
  sul viso scempiato\\
  del torturatore colpevole\\
  in questa stanza.\\
}
\section{Karne}
Isola a nord prevalentemente umana.
\subsection{Citt\`a di Karne}
\subsubsection{Beabury}
\subsubsection{Whystone}
\subsubsection{Blasingdell}
\subsection{Luoghi di Karne}
\subsubsection{Stone Tooth}
\section{Estor}
\subsection{Citt\`a di Estor}
\subsubsection{Artis}
Artis \`e una citt\`a rinomata per il commercio di sostanze magiche e
alchemiche. \`E facile, inoltre, incontrare maghi ed alchimisti, non solo
\begin{quotebox}
  Appena arrivati i pg trovano un grande disordine nel mercato, dovuto a un recente attacco di
  una banda di ladri venuta da fuori dalla citt\`a. L'attacco sembra qualcosa di pi\`u
  importante di quello che \`e successo.
\end{quotebox}
\begin{commentbox}{Storia per Alex}
  hai iniziato ad indagare sull'isola per tuoi motivi o comunque per motivi legati ai forti legami che essa ha con la natura, legami planari etc etc. quello che hai scoperto non e` tanto rilevante all'isola quanto alle persone che vi hai incontrato: non eri solo nell'isola. infatti, una volta arrivato e iniziato a studiarla, una nave dalla forma affusolata e dalle vele molto alte ha approdato ed e` rimasta per giorni attraccata al largo. portava degli stemmi ma li ha tolti subito dopo essere attraccata e quindi non li hai visti. \\
  qualche giorno dopo essere rimasta ferma la nave ha fatto scendere il suo equipaggio a terra. sono scesi principalmente soldati e qualche studioso: portavano stemmi sul petto con disegni simili stilisticamente a raffigurazioni elfiche. in particolare in uno stemma era chiaramente raffigurato qualcosa di simile a un viso di un uomo e di un drago rosso. le cose si sono complicate in fretta , visti i loro modi bellicosi, sei stato catturato e imprigionato nella loro nave. \\
  la nave e` stata ferma per un po' di giorni poi e` ripartita. da quello che hai potuto sentire dalla tua cella l'isola non era quella che cercavano. hanno inoltre parlato di molti uomini che si sono avventurati nelle grotte e nella foresta di notte e non hanno fatto ritorno. ti sei inoltre reso conto di una grande eruzione durante la tua prigionia che pero` non ha messo in crisi la nave essendo approdata al largo se non per un po' di puzzo di zolfo e di cenere. sei inoltre riuscito a capire che quello che cercavano i soldati era qualcosa di antico e legato al fuoco, per questo si erano avventurati all'interno di un vulcano. hanno paura delle reazioni cattive che potrebbe avere il loro datore di lavoro visto il fallimento. la luce del sole la hai rivista dopo uina settiana di digiuno dal cibo a Meyrador nella quale citta` sei stato venduto come schiavo. \\
  Una volta venduto a un mercante nanico, Bhartor, questo e` salpato verso sud nella citta` di Artis: Questo mercante ti ha trattato molto meglio dei precedenti e percio` sei riuscito a recuperare le forze mangiando qualcosa e dormendo un po'. sei anche riuscito a concentrarti e scappare dal tuo nuovo carceriere. \\
  Una volta nella citta` di artis c'e` stato poco tempo prima di venire inseguito e catturato nuovamente: nel sentire il tuo blaterare per spiegargli la tua lunga storia il mercante ha deciso di punirti pubblicamente secondo la legge della citta`. al momento sei legato a un palo, pronto per essere spogliato e frustato nella pubblica piazza.
\end{commentbox}
\begin{quotebox}
 Bhartor \`e disposto ad interrompere la punizione per 10000 gold.
\end{quotebox}
\begin{quotebox}
  Secondo Arwebrias e Athuith c'\`e un contatto collegato a informazioni sulle
  ombre del sole che vive sulle montagne che portano alla citt\`a di Mithlondul.
  Qui sperano di trovare maggiori infomazioni su Kaligtibur.
\end{quotebox}
\begin{commentbox}{Atoria per Christian}
  Ainwen conosce l'identit\`a di Fleiinhar tra la folla. Si trova nella citt\`a
  gi\`a da un po' di giorni e ha avuto problemi con i briganti che gli hanno fatto
  perdere un bel po' di soldi grazie ai loro traffici illeciti.
\end{commentbox}
\begin{quotebox}
  Se catturati sanno perch\'e Alan ha preso accordi con Blasingdell: sono infatti stati
  contattati da egli per un secondo attacco alla citt\`a. Alan ha parlato con il braccio
  destro di Miles Berrick. Miles \`e ammalato e pochi lo sanno: viene tradito dal
  suo pi\`u vicino. Il traditore ha accordi con il governatore di Riggs al quale
  ha promesso di sposare la figlia. Pare che anche Aebeard Verne sia connesso a tutta la
  storia.
\end{quotebox}

\subsubsection{Estor}
Estor \`e una della quattro maggiori isole nel mondo di Halaia. \`E abitata
principalmente da elfi anche se nelle capitali la popolazione \`e omogenea.
L'isola \`e suddivisa da una catena montuosa in una parte nord e sud: Estor
\footnote{Estor elfico per grande regno}
e Umberul\footnote{Elfico per regno notturno}. La capitale si trova a nord
ed \`e chiamata Mithlandris\footnote{Principe che corre: nota che principe in elfico
non ha genere maschile ma \`e abbinato sia ai principi che alle principesse}; a sud
vi \`e Mithlandor\footnote{Principe della notte}, citt\`a gemella
della capitale e principale capoluogo della regione.
Altre due citt\`a sono molto importanti: Menyamar e Meyrador
\footnote{La radice della parola fa rifermento al fratello minore del
  principe, per cui non esiste un equivalente nella lingua comune.
  Rispettivamente possono essere tradotte come principino delle montagne
  e principino della notte}.\\
Niirith Omlir, sorella di Fahlsias Omlir, governa il regno Mithlandris, mentre
il fratello minore governa il regno inferiore.
Niirith \`e sposata con Throdiore Naemaris, con il quale ha due figli. Il pi\`u
grande, Alagor, governa la citt\`a di Meyrador insieme alla sua consorte Elmilin.
La minore si chiama Minireth ed \`e una guaritrice famosa nel regno.\\
Fahlsias \`e sposato con Alotel Adanodel: Nelalwe \`e la figlia maggiore della
giovane coppia elfica. Poco si sa sul conto di Nelalwe, il suo animo avventuroso la
ha spinta a viaggiare da sempre: ha infatti abbandonato la casa del padre poco
dopo aver raggiunto la sua maggiore et\`a.
Indithrel, secondogenita gemella di Nelalwe, regge la citt\`a di Menyamar insieme
al suo consorte Lebrandil, un elfo del basso ceto che per\`o ha dimostrato grandi
doti militari.
Findore \`e il fratello pi\`u piccolo. \`E ossessionato dalla scomparsa della sorella
Nelalwe e vuole seguirla non appena raggiunta la maggiore et\`a.\\
Il governo elfico \`e spesso affiancato da famiglie importanti che danno consiglio
ai regnanti. Delegal Ceruleadel e Elen aiutano i sovrani di Mithlandris nella loro
reggenza. I tre figli di questa coppia sono ormai famosi nelle isole per essere
rispettivamente una brava commerciante, una alchimista potente e un astronomo:
sono Vadania, Liia e Amiel. Amiel \`e famoso anche per la sua promessa sposa, un'umana.\\
Il trono di Mithlandor ha come consiglieri Felosial Guernodel e Deroua Galothel.
Questi non hanno figli.


\begin{figure}[h]
\centering
\includegraphics[width=0.5\textwidth]{umberulmap.png}
\caption{Mappa di Mithlandor}
\end{figure}


\subsubsection{Mithlandor}\label{cyt:mithlandor}
Mithlandor \`e la seconda capitale per importanza del regno elfico. Non ha
un'area urbana molto estesa ma ha una densit\`a di popolazione superiore anche alla
capitale. Essa \`e situata al centro del lago \hyperref[loc:fehereiah]{Fehereiah}, discendente dal fiume
\hyperref[loc:direl]{Direl} Proveniente dal \hyperref[loc:deinmer]{Deinmer} che origina dai monti
\hyperref[loc:nerei]{Nerei} per poi buttarsi in mare sulla costa orientale del continente.
Anticamente il lago su cui affiora la citt\`a di Mithlandor era un vulcano che
si \`e fatto strada lentamente tra le montagne generando fondamenta solide per la
secondogenita citt\`a elfica.\\
Mithlandor ha una struttura esagonale per la grande solidit\`a del numero sei. Le fortificazioni
sono minime vista la posizione strategica nel territorio e l'isolamento del lago. In periodi
di carestia la citt\`a ha riserve sufficienti per un anno intero. Sebbene l'isolamento la
citt\`a comunica con le torri e con l'esterno attraverso vie sotterranee e subacquee e non corre il
rischio di allagamento trovandosi al di sopra del bacino. Le vie sotterranee sono segretissime e sono in
pochi a conoscerne l'esistenza.\\
Il centro cittadino \`e il cuore della citt\`a, li vi risiede il mercato e tutti gli edifici adibiti
a ruoli politici o economici.
Non essendo dotata di un palazzo reale una delle sei torri che emergono dal lago \`e usata per copi politici
e, le riunioni si tengono all'ultimo piano della torre. Le torri sono di forma particolare:
Di pianta quadrata si innalzano matenendo la stuttura di un obelisco ma arricciata su se stesso
per un quarto di giro. Questa struttura da alla torre una maggiore elasticit\`a necessaria
poich\`e immersa in acqua.
\begin{commentbox}{La leggenda di Kraul: la Tempesta Rossa}
  Nell'era della prima venuta degli elfi in quelle terre era da poco cessata una grande
  carestia. Il periodo rigoglioso che port\`o la razza elfica a prosperare era nato da
  una grande sofferenza che aveva colpito la terra tempo prima.
  Kraul comandava quelle terre prima di loro:
  era una creatura spregevole e avida, il suo potere era grande e nessuno,
  all'interno della regione, viveva libero bens\`i sotto il suo dominio.
  In quel tempo, sebbene il dominio di Kraul creasse grande sofferenza alla terra,
  la natura cresceva rigogliosa, armonizzando le sue forme attorno alla vita che risiedeva
  in quelle pigre colline che adesso sono i Monti Nerei.
  Il drago si stagliava nel cielo per controllare il suo territorio non pi\`u alto di un
  uccello dal mare. Il calore emanato dal suo corpo rendeva il territorio al suo passaggio
  orridamente caldo: uno scirocco fetido che era in grado di scioglere le poche nevi che
  avvolgevano all'epoca le colline.\\
  Faxut, un gigante delle nubi che aveva attraversato in mare, decise di prendere dimora
  non molto lontano dal dominio del drago. Si guard\`o bene dall'entrare nei territori di
  Kraul, non perch\`e lo temesse, ma perch\`e non era di indole bellicosa, e preferiva la
  pace ai litigi. Kraul non era un essere dagli stessi principi: una volta saputo
  dell'arrivo del gigante, senza pensarci due volte, decise di espandere arrogantemente
  il suo dominio e scacciare cos\`i Faxut.
  Sentendosi provocato il gigante non si lasci\`o certo sgominare dal rettile e lo
  rimprover\`o, ma senza minacciarlo, cercando a tutti i costi di mantenere la pace
  e di evitare uno scontro.
  Kraul si sent\`i offeso e decise di far pagare al gigante cara la sua intraprendenza
  nello sfidarlo cos\`i apertamente.\\
  Una notte, tent\`o furtivamente di avvicinarsi
  a Faxut ma non si rese conto di essere stato scoperto, e si dovette ritirare.
  Il giorno seguente il gigante si rec\`o dal drago, deciso a difendere la sua posizione.
  Sebbene Kraul fosse incolpato giustamente dal gigante, neg\`o i fatti: us\`o tutta
  la sua falsa gentilezza per congedare il gigante, serbando nel petto fiame nere di odio e
  rosse di rabbia.\\
  Col passare dei giorni, l'atteggiamento di Kraul divent\`o sempre pi\`u
  provocatorio e Faxut si inaspr\`i a tal punto da prendere una decisione drastica.
  Non si sa fino a che punto la meschinit\`a del drago fosse penetrata nel cuore del
  gigante, ma lo corruppe a tal punto che Faxut cominci\`o a desiderare,
  sempre di pi\`u, la vita del drago.\\
  Riusc\`i a sfruttare il territorio a suo vantaggio e, quando il momento fu propizio,
  richiam\`o le forze del cielo cos\`i, con tuoni, fulmini e tutta la sua forza e quelle dei venti,
  fece crollare la cima di una ripida collina sul drago, colto in trappola per l'occasione.
  Kraul Si sent\`i soffocare, schiacciato dal peso delle rocce. Era stato ingannato:
  per la prima volta era stato messo in ginocchio e per di pi\`u con l'inganno.
  Sebbene Kraul fosse sempre stato un essere spregevole, falso e vigliacco,
  si trov\`o sconfitto dalle sue stesse armi e non fu cos\`i ingenuo da non capirlo.
  Fece ricorso a tutta la sua ira.
  Le fiamme infernali che balenarono dal profondo del suo cuore divorarono le
  rocce sovrastanti. Pezzi di terra fusa cominciarono a mescolarsi con i metalli del
  sottosuolo in un caos liquefatto che cominci\`o presto a sgorgare dalla cime della collina
  crollata sopra la sua testa ed infine  il drago risal\`i il flusso di rocce infuocate
  che aveva creato e sorse, dapprima con la sua testa squamosa e poi con tutto il corpo,
  dal culmine delle rocce. Il drago non risparmi\`o le forze, pur sapendo
  che sarebbe stata una azione disperata: l'odio verso Faxut era tale da andare oltre
  tutto il resto. Il gigante, era scuro in viso e nell'animo quando vide il rettile
  emergere dalla montagna. L'odio che lo aveva accecato si trasform\`o in paura facendogli
  vedere, dapprima chiaramente e poi sempre pi\`u indistintamente, ci\`o che stava succedendo.
  I sentimenti del gigante si mescolarono nel cielo che si mise a piangere su di lui,
  conscio di ci\`o che stava per succedere. In preda all'isteria del momento, il gigante
  si scagli\`o sul drago, insieme a tutta la volta di nuvole, deciso a ucciderlo. Lo stesso
  fece il drago, che spicc\`o il volo con le sue ultime forze. La terra trem\`o allo scontro
  dei due e si intromise alzando la voce e scuotendosi pi\`u forte che poteva ma
  non riusc\`i a farsi sentire da i due che ormai avevano infuocato il cielo.
  La terra url\`o ancora pi\`u forte, le colline si rizzarono in tutta la loro altezza sulla
  schiena della terra intirizzita dalla rabbia  ma fu troppo tardi:
  quello che le cime, ormai alte, videro ai loro piedi era il
  raccapricciante spettacolo della fine. La luce del fuoco aveva trasformato il cielo notturno
  in un'alba placida che adesso avvolgeva le montagne. Il corpo del gigante stava riverso sulla pancia
  comprendo completamente il drago, schiacciato dalle sue membra. L'acqua continuava a scendere
  dal cielo senza tregua e iniziava a riempire il bacino creato dalla furia di Kraul
  sommergendo a poco a poco prima le mani poi i piedi di Faxut.
  Il cielo pianse ininterrottamente lacrime amare, tanto amare da
  sciogliere la roccia intorno a loro. Le nuove montagne divennero,cos\`i, aguzze e cupe.
  Col passare del tempo le ferite della terra furono sanate ma, il paesasggio, non
  somigliava pi\`u a quello che era una volta. Le dolci colline ora erano diventate
  montagne scure, aguzze e altissime; in mezzo a loro un lago con una grande isola al centro
  si mostrava alla valle.
  Faxut e Kraul riposano in fondo al lago ma il cuore del drago pulsa ancora sotto la
  montagna poich\`e l'errore pi\`u grande del gigante fu quello di desiderare di spegnerlo:
  un cuore incendiato non si pu\`o spegnere con le fiamme, e la pioggia su di esso non \`e
  riuscita a soffocare la brace che ancora riscalda le acque del lago di \hyperref[loc:fehereiah]{Fehereiah}.
  
\end{commentbox}
\`E una citt\`a famosa per le pratiche monastiche e psioniche, si potrebbe definire
una culla della cultura di queste due discipline.
\begin{commentbox}{Attentato}
  Per mano di Kaligtibur si \`e compiuto un attentato alla vita dei consiglieri della
  famiglia reale regnante a Mithlandor. La colpa non \`e stata ancora data, ci sono
  indagini in corso in citt\`a e i controlli sono molto elevati. Sebbene non sia di
  mentalit\`a razzista, questo avvenimento ha contribuito non poco ad alzare
  i sospetti contro i non elfi.
\end{commentbox}
\begin{commentbox}{Miniera}
  La miniera pare che nasconda dentro di se il cuore di Kraul: un antico artefatto che
  pare permetta di controllare i draghi. Il cuore di Kraul \`e in realt\`a un nome
  locale che si riferisce ad un artefatto molto pi\`u generale chiamato Anima del
  Fuoco che ha origine da grandi fonti di fiamme e calore. Pare che dia il potere di
  controllo del fuoco e chiss\`a quali altri poteri. C'\`e chi dice che dia
  il potere di controllare un drago rosso.
\end{commentbox}
\subsection{Luoghi di Estor}
\subsubsection{Cerchio Dei Gabbiani}\label{loc:gabbiani}
Zona in cima alla collina poco lontana da Artis: zona molto anctica ma ritenuta
malfamata per via di un gruppetto di briganti chiamati dal popolo ``i pirati
di terra''.
\subsubsection{Lago Fehereiah}\label{loc:fehereiah}
Lago su cui sorge \hyperref[loc:mithlandor]{Mithlandor}
\subsubsection{Fiume Deinmer}\label{loc:deinmer}
Fiume che ha le sue origini dai chiacciai perenni dei monti Nerei. Da esso
deriva il Direl.
\subsubsection{Fiume Direl}\label{loc:direl}
Fiume proveniente dal Deinmer che origina il lago Fehereiah.
\subsubsection{Monti Nerei}\label{loc:nerei}
Catena montuosa principale dell'isola di estor.
\subsubsection{Miniera di Ster}\label{loc:ster}
Miniera sulle montagne a est di Mithlandor
\section{Personaggi}
\subsection{Ulfe}\label{char:ulfe}
Ogre capitoano del manipolo di orchi dentro lo stone tooth.
\subsection{Gerey Horne}
Uno dei capi della guardia di Blasingdell. Amico di Undecimus.
\subsubsection{Cenbert, Hele}\label{char:cenbert}
Guardie di Blasingdell amiche di Undecimus.
\subsection{Herumbr, Alin, Linda}\label{char:herumbr}
Tre nomi misteriosi di possibili halfling trovati all'interno
dello stone tooth con un contratto da parte di Aebeard V.
Trovati due corpi su tre.
\subsection{Sige}\label{char:sige}
Locandiere di Whyam. Da il contratto delle lame di Durgeddin.
\subsection{Wol}\label{char:wol}
Locandiere di Blasingdell. In stretto contatto con Sige e Arwebrias
\subsection{Sir Miles Berrick}\label{char:milesberrick}
Reggente di Blasingdell, pare sia un sovrano giusto e attento alla gente
seppur molto preso da questioni personali ed economiche.
\subsection{Snurrevin}\label{char:snurrevin}
Duergar superstite dal raid dello stone tooth. Catturato dalle guardie di
Blasingdell dipo la sua fuga dalla montagna \`e stato imprigionato in una
nave piena di pirati e mercenari per essere interrogato. Pare che le
sue informazioni riguardo alla posizione strategica della montagna
potessero fare gola a molte bocche.
\subsubsection{Brawngnaw}\label{char:brawngnaw}
Fedele topo di Snurrevin, assassinato malamente davanti agli occhi del padrone.
\subsection{Bryaney Pethey}\label{char:bryaney}
Comandante della guardia cittadina di Blasingdell. Uomo compromesso. Si dice che
sia un uomo giusto ma sia troppo coinvolto dal suo lavoro e la sua moralit\`a
sia per questo motivo compromessa. I pareri su di lui sono discordanti. Geoffrey
Lo accusa, Gerey lo difende.
\subsection{Liamond}\label{char:liamond}
Ricercato nel quartiere mercantile di Blasingdell come un capo mafia di una
lunga serie di furtarelli di citt\`a.
\subsection{Shiishtel}\label{char:shiishtel}
Spietato combattente di arene. Yuan ti che combatte per fama e denaro, rispetta
solo le persone che secondo lui possono meritarsi la sua attenzione. Ha risparmiato una
sola volta no sfidante nella sua vita: Ralphia.
\subsection{Ralphia}\label{char:ralphia}
Conosce Arwebrias, Risparmiato da Shiistel.
\subsection{Aebeard Verne}\label{char:aebeard}
Misterioso Commerciante della zona del porto. Era stato indicato da Gerey come un
aiuto per risolvere il mistero della nave vuota. Fuori citt\`a quando \`e stato cercato.
\subsection{Capitano Alan: 12 anime}\label{char:alan}
Famoso pirata e predone dei mari del sud. Responsabile anche del rapimento di Nanande e
del massacro dei suoi compagni. Mercenario ambizioso e senza scrupoli assoldato
dalla guardia cittadina per sorvegliare Snurrevin. Collegato in qualche modo alle
Ombre Del Sole.
\subsubsection{Gorbal Tuhurk}\label{char:gorbal}
Nostromo e fedele braccio destro di Alan. 
\subsubsection{Barion Gilli Strongbeard}\label{char:barion}
Mercenario di provenienza ignota al servizio di Alan.
\subsubsection{Ezzel'n Shii}\label{char:ezzel}
Vedetta di Alan.
\subsection{Geoffrey/Athuith}\label{char:geoffrey}
Mercante di blasingdel e trafficante al porto, conosciuto con un doppio nome.
Sembre che abbia interessi simili a quelli di Arwebrias anche se si occupano di
cose diverse.
\subsection{Arwebrias Galothel}\label{char:arwebrias}
Elfa ricercatissima in citt\`a per la sua fama dovuta probabilmente alla sua identit\`a
molto ben celata e alla sua difficolt\`a nel lasciare tracce o di farsi catturare.
Pare che sappia molte pi\`u cose di quelle che dice. Sa molte cose di politica e difende
degli ideali di libert\`a. Non le piace Athuith ma lo rispetta come fuorilegge. \`E al
momento collegata con tutti gli incarichi svolti finora, o come mandante o per complicit\`a
e coinvolgimento.
\subsection{Quel' Cha}
\label{char:quelcha}
\subsection{Bhartor}\label{char:bhartor}
Mercante di schiavi nanico residente ad Artis
\subsection{Sieg, Paltronur e Fleiinhar}\label{char:sieg}
Tre predoni in contatto con \hyperref[char:alan]{Alan}. Hanno ricevuto da lui un biglietto con scritto: \\
\textit{
  Tenetevi pronti: vi chiamer\`o in seguito. \\ \\Alan}
Non interessano a molte persone perch\`e sono semplici pirati ma hanno una taglia
sulle loro teste di 1000gp vivi e 500gp morti.
\subsection{Punie}\label{char:punie}
Mezz'elfo originario di Meryador. Si \`e ritirato sulle motagne per via della sua indole
schiva in seguito ad un disaccordo interno all cerchia elfica dopo l'ingresso tra
i ranghi di Kaligtibur.
La sua casa \`e un rifugio per molti viandanti: tiene anche a bada molti dei mostri e
predoni in zona.
\`E molto diffidente: si fida delle persone grazie al suo istinto, a pelle e difficilmente
cambia opinione
\begin{quotebox}
  Per vedere chi gli sta simpatico si fa un DC CHAR 15. ogni volta che una persona che non gli sta
  simpatica tenta di convincerlo deve fare un DC CHAR 20. Se fallisce il tiro diventa 17.
\end{quotebox}
\begin{commentbox}{Appena incontrato Punie}
Punie ha catturato uno gnoll chiamato Hetkir, figlio di Kaharthak, per fare uno scambio.
Una tregua in cambio della vita del figlio: la zona deve essere abbandonata dai saccheggi di
Kaharthak. Sicuramente la proposta \`e presa in considerazione poich\'e la vita
del figlio \`e molto importante. Kaharthak vuole per\`o essere sicuro sia che suo figlio stia
bene sia che Punie stia bene. In caso si accorgesse che uno dei due \`e morto o in fin di vita
non tarderebbe ad attaccare la casa di Punie. Punie \`e gravemente ferito dal combattimento
con Hetkir al momento sta sopravvivendo con le scorte rimaste. La montagna \`e piena di scorribande.
\end{commentbox}
\subsection{Kaharthak e Hetkir}\label{char:kaharthak}
Gnoll che comanda le brigate sulle montagne. Famoso predone conosciuto per essere
un leader carismatico e influente anche su altre razze quali gli orchi.
Hetkir \`e suo figlio e succeessore. Kaharthak tiene a lui pi\`u di ogni altra cosa
essendo l'ultimo membro della famiglia rimastogli
\subsubsection{Weiss e Fleart}\label{char:weiss}
Lupi delle nevi fedelissimi a Kaharthak
\subsubsection{Kraul e Faxut}\label{char:kraul}
Gigante e drago nella leggenda

\end{document}