\begin{commentbox}{La leggenda di Kraul: la Tempesta Rossa}
  Nell'era della prima venuta degli elfi in quelle terre era da poco cessata una grande
  carestia. Il periodo rigoglioso che port\`o la razza elfica a prosperare era nato da
  una grande sofferenza che aveva colpito la terra tempo prima.
  Kraul comandava quelle terre prima di loro:
  era una creatura spregevole e avida, il suo potere era grande e nessuno,
  all'interno della regione, viveva libero bens\`i sotto il suo dominio.
  In quel tempo, sebbene il dominio di Kraul creasse grande sofferenza alla terra,
  la natura cresceva rigogliosa, armonizzando le sue forme attorno alla vita che risiedeva
  in quelle pigre colline che adesso sono i Monti Nerei.
  Il drago si stagliava nel cielo per controllare il suo territorio non pi\`u alto di un
  uccello dal mare. Il calore emanato dal suo corpo rendeva il territorio al suo passaggio
  orridamente caldo: uno scirocco fetido che era in grado di scioglere le poche nevi che
  avvolgevano all'epoca le colline.\\
  Faxut, un gigante delle nubi che aveva attraversato in mare, decise di prendere dimora
  non molto lontano dal dominio del drago. Si guard\`o bene dall'entrare nei territori di
  Kraul, non perch\`e lo temesse, ma perch\`e non era di indole bellicosa, e preferiva la
  pace ai litigi. Kraul non era un essere dagli stessi principi: una volta saputo
  dell'arrivo del gigante, senza pensarci due volte, decise di espandere arrogantemente
  il suo dominio e scacciare cos\`i Faxut.
  Sentendosi provocato il gigante non si lasci\`o certo sgominare dal rettile e lo
  rimprover\`o, ma senza minacciarlo, cercando a tutti i costi di mantenere la pace
  e di evitare uno scontro.
  Kraul si sent\`i offeso e decise di far pagare al gigante cara la sua intraprendenza
  nello sfidarlo cos\`i apertamente.\\
  Una notte, tent\`o furtivamente di avvicinarsi
  a Faxut ma non si rese conto di essere stato scoperto, e si dovette ritirare.
  Il giorno seguente il gigante si rec\`o dal drago, deciso a difendere la sua posizione.
  Sebbene Kraul fosse incolpato giustamente dal gigante, neg\`o i fatti: us\`o tutta
  la sua falsa gentilezza per congedare il gigante, serbando nel petto fiame nere di odio e
  rosse di rabbia.\\
  Col passare dei giorni, l'atteggiamento di Kraul divent\`o sempre pi\`u
  provocatorio e Faxut si inaspr\`i a tal punto da prendere una decisione drastica.
  Non si sa fino a che punto la meschinit\`a del drago fosse penetrata nel cuore del
  gigante, ma lo corruppe a tal punto che Faxut cominci\`o a desiderare,
  sempre di pi\`u, la vita del drago.\\
  Riusc\`i a sfruttare il territorio a suo vantaggio e, quando il momento fu propizio,
  richiam\`o le forze del cielo cos\`i, con tuoni, fulmini e tutta la sua forza e quelle dei venti,
  fece crollare la cima di una ripida collina sul drago, colto in trappola per l'occasione.
  Kraul Si sent\`i soffocare, schiacciato dal peso delle rocce. Era stato ingannato:
  per la prima volta era stato messo in ginocchio e per di pi\`u con l'inganno.
  Sebbene Kraul fosse sempre stato un essere spregevole, falso e vigliacco,
  si trov\`o sconfitto dalle sue stesse armi e non fu cos\`i ingenuo da non capirlo.
  Fece ricorso a tutta la sua ira.
  Le fiamme infernali che balenarono dal profondo del suo cuore divorarono le
  rocce sovrastanti. Pezzi di terra fusa cominciarono a mescolarsi con i metalli del
  sottosuolo in un caos liquefatto che cominci\`o presto a sgorgare dalla cime della collina
  crollata sopra la sua testa ed infine  il drago risal\`i il flusso di rocce infuocate
  che aveva creato e sorse, dapprima con la sua testa squamosa e poi con tutto il corpo,
  dal culmine delle rocce. Il drago non risparmi\`o le forze, pur sapendo
  che sarebbe stata una azione disperata: l'odio verso Faxut era tale da andare oltre
  tutto il resto. Il gigante, era scuro in viso e nell'animo quando vide il rettile
  emergere dalla montagna. L'odio che lo aveva accecato si trasform\`o in paura facendogli
  vedere, dapprima chiaramente e poi sempre pi\`u indistintamente, ci\`o che stava succedendo.
  I sentimenti del gigante si mescolarono nel cielo che si mise a piangere su di lui,
  conscio di ci\`o che stava per succedere. In preda all'isteria del momento, il gigante
  si scagli\`o sul drago, insieme a tutta la volta di nuvole, deciso a ucciderlo. Lo stesso
  fece il drago, che spicc\`o il volo con le sue ultime forze. La terra trem\`o allo scontro
  dei due e si intromise alzando la voce e scuotendosi pi\`u forte che poteva ma
  non riusc\`i a farsi sentire da i due che ormai avevano infuocato il cielo.
  La terra url\`o ancora pi\`u forte, le colline si rizzarono in tutta la loro altezza sulla
  schiena della terra intirizzita dalla rabbia  ma fu troppo tardi:
  quello che le cime, ormai alte, videro ai loro piedi era il
  raccapricciante spettacolo della fine. La luce del fuoco aveva trasformato il cielo notturno
  in un'alba placida che adesso avvolgeva le montagne. Il corpo del gigante stava riverso sulla pancia
  comprendo completamente il drago, schiacciato dalle sue membra. L'acqua continuava a scendere
  dal cielo senza tregua e iniziava a riempire il bacino creato dalla furia di Kraul
  sommergendo a poco a poco prima le mani poi i piedi di Faxut.
  Il cielo pianse ininterrottamente lacrime amare, tanto amare da
  sciogliere la roccia intorno a loro. Le nuove montagne divennero,cos\`i, aguzze e cupe.
  Col passare del tempo le ferite della terra furono sanate ma, il paesasggio, non
  somigliava pi\`u a quello che era una volta. Le dolci colline ora erano diventate
  montagne scure, aguzze e altissime; in mezzo a loro un lago con una grande isola al centro
  si mostrava alla valle.
  Faxut e Kraul riposano in fondo al lago ma il cuore del drago pulsa ancora sotto la
  montagna poich\`e l'errore pi\`u grande del gigante fu quello di desiderare di spegnerlo:
  un cuore incendiato non si pu\`o spegnere con le fiamme, e la pioggia su di esso non \`e
  riuscita a soffocare la brace che ancora riscalda le acque del lago di \hyperref[loc:fehereiah]{Fehereiah}.

\end{commentbox}