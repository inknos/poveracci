
\subsubsection{Mithlandor}\label{cyt:mithlandor}
Mithlandor \`e la seconda capitale per importanza del regno elfico. Non ha
un'area urbana molto estesa ma ha una densit\`a di popolazione superiore anche alla
capitale. Essa \`e situata al centro del lago \hyperref[loc:fehereiah]{Fehereiah}, discendente dal fiume
\hyperref[loc:direl]{Direl} Proveniente dal \hyperref[loc:deinmer]{Deinmer} che origina dai monti
\hyperref[loc:nerei]{Nerei} per poi buttarsi in mare sulla costa orientale del continente.
Anticamente il lago su cui affiora la citt\`a di Mithlandor era un vulcano che
si \`e fatto strada lentamente tra le montagne generando fondamenta solide per la
secondogenita citt\`a elfica.\\
Mithlandor ha una struttura esagonale per la grande solidit\`a del numero sei. Le fortificazioni
sono minime vista la posizione strategica nel territorio e l'isolamento del lago. In periodi
di carestia la citt\`a ha riserve sufficienti per un anno intero. Sebbene l'isolamento la
citt\`a comunica con le torri e con l'esterno attraverso vie sotterranee e subacquee e non corre il
rischio di allagamento trovandosi al di sopra del bacino. Le vie sotterranee sono segretissime e sono in
pochi a conoscerne l'esistenza.\\
Il centro cittadino \`e il cuore della citt\`a, li vi risiede il mercato e tutti gli edifici adibiti
a ruoli politici o economici.
Non essendo dotata di un palazzo reale una delle sei torri che emergono dal lago \`e usata per copi politici
e, le riunioni si tengono all'ultimo piano della torre. Le torri sono di forma particolare:
Di pianta quadrata si innalzano matenendo la stuttura di un obelisco ma arricciata su se stesso
per un quarto di giro. Questa struttura da alla torre una maggiore elasticit\`a necessaria
poich\`e immersa in acqua.
\`E una citt\`a famosa per le pratiche monastiche e psioniche, si potrebbe definire
una culla della cultura di queste due discipline.\\
\begin{figure}[h]
  \centering
  \includegraphics[width=0.5\textwidth]{umberulmap.png}
  \caption{Mappa di Mithlandor}
\end{figure}