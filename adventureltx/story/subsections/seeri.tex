\subsection{Seeri}
Seeri \`e quella che al giorno d'oggi verrebbe definta la capitale del regno.
La sua caratteristica sono le alte mura di un freddo minerale verde: Peridotite.
L'insediamento, ormai grande, di Seeri \`e popolato quasi esclusivamente da una unica etnia:
gli Aliseri o altrimenti detti popolo delle nuvole.
Ad arricchire quest'ultima, gli Eiredin, vengono considerati mezzo sangue
dagli Aliseri: Eiredin significa popolo dal mare.
\begin{commentbox}{Culto degli Aliseri}
  Il culto degli aliseri \`e strutturato su numeri e materiali preziosi. Il numero 5 \`e
  ricorrente nelle loro preghiere e nelle loro mitologie. Il dio venerato \`e una forza
  della natura chiamato Kheeriiu. Viene raffigurato come un leone avente un occhio chiuso
  e uno aperto, in segno di una veglia sempre attenta e sempre riposante: filosofia di vita,
  anch'essa, ricorrente.
\end{commentbox}
A governare la citt\`a ci sono gli otto saggi, quattro dei quali sono anche i maggiori esponenti
religiosi e, per questo, non sono tenuti a parlare se non con gli altri saggi.\\
Tutti e otto hanno sviluppato grandi poteri magici in una scuola diversa della magia attuale.
Le scuole arcaiche sono Veicolazione, Trasmigrazione e Mutazione.
Di questi otto saggi, Felahia \`e il pi\`u noto, sia per la sua capacit\`a di guida
del popolo, sia per le sue grandi dote oratorie.
Il maestro di mutazione ha un'arma in serbo, un golem di peridotite che sta preparando
in caso 
\begin{commentbox}{5}
  Il numero 5 \`e raffigurativo del quinto elemento. I primi quattro hanno un corrispettivo
  Nelle credenze odierne: Freddo, Caldo, Spirito, Consistenza e Melinka.
  Melinka \`e un nome intraducibile ed \`e l'elemento pi\`u importante ma che deve esistere
  in minore quantit\`a rispetto agli altri poich\`e \`e molto pi\`u potente di essi.
  Melinka significa tutto ci\`o che \`e incomprensibile: \`e anche una parola del
  linguaggio che descrive le cose che attraggono pur essendo ignote.
\end{commentbox}
Gli Aliseri sono in queste terre da non molto. Hanno compiuto un grande esodo dal mare,
fondendosi insieme agli Eiredin. Sebbene gli Eiredin vengano spesso trattati con poco
rispetto c'\`e da dire che hanno caratteristiche pi\`u uniche che rare rispetto agli Aliseri.
Pi\`u di una volta nella storia un Eiredin si \`e distinto nella storia degli Aliseri
Per qualche sua truvata non convenzionale, che ha spesso aiutato ad uscire da qualche
brutta situazione. Malgrado questo, gli aliseri tendono comunque a sentirsi superiori, poich\'e
non sentono mescolanze nelle loro vene.
\begin{commentbox}{Lingua parlata}
  La lingua in uso \`e il draconico sebbene i draghi siano esseri temuti e rispettati.
  Gli aliseri hanno da sempre parlato il draconico. Gli Eiredin lo hanno imparato da essi.
  Sebbene il popolo sia in cerca di una propria identit\`a la lingua mantenuta \`e questa
  per una questione di praticit\`a magica.
\end{commentbox}